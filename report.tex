\documentclass[a4paper,11pt]{article}
\usepackage[T1]{fontenc}
\usepackage[utf8]{inputenc}
\usepackage{lmodern}
\usepackage{xspace}
\usepackage{latexsym}
\usepackage{epsfig}
\usepackage{verbatim}
\usepackage{shadow}
\usepackage{amssymb}
\usepackage{amsmath}
\usepackage{algorithm}
\usepackage{mathtools}
\usepackage{graphicx}
\usepackage{color}
\usepackage{bm}
\usepackage{fancyhdr}

\addtolength{\textwidth}{4cm}
\addtolength{\textheight}{4cm}
\addtolength{\oddsidemargin}{-2cm}
\addtolength{\topmargin}{-2.5cm}

\def\cc#1{\mathsf{#1}}
\def\CLS{\ensuremath{\cc{CLS}}\xspace}
\def\PPAD{\ensuremath{\cc{PPAD}}\xspace}

\def\problem#1{\textsc{#1}}
\def\EOL{\problem{EndOfLine}\xspace}
\def\EOML{\problem{EndOfMeteredLine}\xspace}
\def\CC{Colorful Carath\'eodory }
\def\org{\bm{0}}
\def\v{\textbf{v}}
\def\CCP{\problem{ColorfulCarath\'eodory}\xspace}

\title{IE 598 JG Games, Markets and Mathematical Programming \\ \CCP $in$ \PPAD \\ A simpler proof using LCPs}

\author{
	{\sc Rucha Kulkarni} \\
	\texttt{ruchark2@illinois.edu}
	\and
	{\sc Vasilis Livanos} \\
	\texttt{livanos3@illinois.edu}
}

\date{}

\begin{document} \maketitle

\section{Introduction}

%FIXME: Rucha. This is simply the proposal copy-pasted and needs lots of work.

\par The \CC theorem is a classical result in graph theory. The theorem states that given $d+1$ sets
$C^i, i \in [d+1]$ of $d+1$ vertices each in $R^d$, so that the origin lies in the convex hull of each of
these sets, there exists a set of $d+1$ vertices $S$, such that exactly one vertex from each set $C^i$
belongs to $S$, and the origin lies in the convex hull of this set.

\par For visualizing the theorem, one can think all vertices of one input set $C^i$ are assigned one color,
which we denote by $i$. Then given $d+1$ monochromatic sets with the origin in the convex hull of each, the
\CC theorem proves the existence of a set of $d+1$ vertices of distinct colors, with the origin in the convex
hull of this set. We call such a set \textit{panchromatic}.

\par The natural computational problem of finding a panchromatic vertex set has been proven to lie in \PPAD
\cite{CCP_PPAD}. We propose to find an alternative simpler proof of membership of the \CC Problem (\CCP) in
\PPAD. The proof idea is to design a Linear Complementarity Program (LCP) for the problem, and prove that if
Lemke's algorithm is implemented on the LCP, the resulting path followed by the algorithm converges. This path,
if finite, is an instance of the characteristic \PPAD-Complete problem \EOL. Thus its a valid reduction from
\CCP to \EOL, proving \CCP to be in \PPAD.

\section{Related Work}

%FIXME: Vasilis

\section{Preliminaries}

%FIXME: Rucha

\subsection{The \CC Theorem}

\subsection{Complexity Classes of Total Functions}

\subsection{Linear Complementarity Problems}

\section{An LCP for \CCP}

%FIXME: Vasilis 
\par The main purpose of this section is to provide an LCP that captures the solutions of \CCP and that is
correct. The existence of such an LCP already is a strong indication that \CCP $\in$ \PPAD. However, this
result requires that the LCP has no secondary rays and Lemke's algorithm can be applied to it, which as we
will see is not the case for our designed LCP.

\par We start off by introducing a set of necessary and sufficient constraints that capture the solutions to
\CCP. Specifically, given $d+1$ sets of $d+1$ vertices where we denote the $j$th vertex of color $i$ by
$\v^i_j$, we consider the coefficients of these vertices in a possible convex combination, and denote them
by $a^i_j$. Consider a set $S$ of vertices that is a solution to \CCP. Then, by the definition of the problem,
we know that

\begin{itemize}
\item At a solution, if one coefficient of a color is strictly positive, then all other coefficients of the same
color are equal to zero. We capture this property by introducing the following complementarity constraint

\[
\forall i, j \qquad a^i_j \geq 0 \qquad \bot \qquad \sum_{k \neq j} {a^i_k} \geq 0
\]

This property assures that at most one coefficient is strictly positive in a solution. Note that we do not force
exactly one coefficient to be positive because of the possibility of $\org$ lying in the boundary of $conv(S)$
which is a subspace of dimension lesser than $d$.

\item At a solution, $\org \in conv(S)$. However, we do not know \textit{a priori} which $a^i_j$ are going to be
non-zero. Therefore, we capture this property by imposing the following equality constraint

\[
\sum_{i = 1}^{d+1} { \sum_{j = 1}^{d+1} {a^i_j \v^i_j } } = \org
\]

\item The $a^i_j$ define a convex combination of $\v^i_j$, therefore we have to impose the following equality
constraint as well

\[
\sum_{i = 1}^{d+1} { \sum_{j = 1}^{d+1} {a^i_j} } = 1
\]

%FIXME: Rucha, Proof of correctness for the correct LCP

\end{itemize}

\section{Existence of a Secondary Ray}

%FIXME: Vasilis

\section{Conclusion}

%FIXME: We'll see who

\bibliographystyle{plain}
\bibliography{report}

\end{document}
